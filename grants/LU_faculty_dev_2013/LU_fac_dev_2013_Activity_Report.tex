\documentclass{article}

\usepackage{booktabs}

\title{Faculty Development Grant Activity Report}
\author{Kenneth Fortino}

\begin{document}
\maketitle


\section{Introduction}

This report documents the activities of my lab that were funded through the Faculty Development Grant awarded in May 2013 by the Committee for Faculty Development and Research and the PVPAA's office. The grant was awarded to provide the equipment required to investigate the dynamics of organic matter processing in small ponds.  Furthermore, the grant made it possible for my lab to train 7 Longwood students who's work resulted in 4 poster presentations at regional and national meetings and an invited talk at Virginia Commonwealth University.



\section{Research Progression}
\subsection{Research Activities}
During the past year my lab has initiated 3 major research projects that were made possible with equipment purchased with this grant.
\subsubsection{Survey of Sediment Organic Matter Content and Biodiversity in Small Ponds}
Beginning in the Spring of 2013 with 2 Longwood students (Carly Martin '13 and Leanna Tacik '14), my lab initiated a survey of the sediment organic matter content and macroinvertebrate biodiversity in several small ponds around Farmville, VA. I continued this work during the Summer of 2013 with Leanna Tacik while participating in the LU-PRISM summer research program. The funding provided by the grant allowed my lab to purchase a microscope used to process the samples, as well as other needed sample processing supplies and reagents. This project provided data that furthered our understanding of the inventory of organic matter in small ponds. These results were included in posters presented at the Sigma Xi Student Research Symposium at Hampden Sydney College, the Cook-Cole Student Research Symposium at Longwood University, the annual meeting of the Association of Southeastern Biologists in Spartanburg, SC, and at the annual meeting of the Society for Freshwater Science in Portland, OR. Furthermore, samples collected during this project are currently being processed in the lab and should yield more insights into the biodiversity of sediment communities in these ponds.

\subsubsection{Quantification of Leaf Litter Decomposition Rate in Small Ponds}
In the Fall of 2013, my lab initiated an experiment to quantify the decomposition rate of leaf litter in small ponds. This project has so far trained 4 Longwood students in the Fall of 2013 (Leanna Tacik '14, Annie Choi '15, Kasey McCusker '14, and Andreas Gregoreau '15), an additional student (Kaitlyn Peters '16) in the Spring of 2014, and another student (DJ Letteri '16) during the Fall of 2014. The funds from the grant allowed the lab to purchase the materials needed to set up this experiment (e.g., litter bags) and process the samples. This project is ongoing but the results that we have obtained to date were included in posters presented at the annual meeting of the Association of Southeastern Biologists in Spartanburg, SC, and at the annual meeting of the Society for Freshwater Science in Portland, OR. This research was also featured at the groundbreaking of the Longwood Environmental Education Center in the Spring of 2014. The samples collected from this experiment continue to be processed by my lab and should continue to yield data that will inform our understanding of the dynamics of leaf litter breakdown in small ponds.

\subsubsection{Effect of Leaf Litter Decomposition on Nutrient Cycling in Small Pond Sediments}
The final project made possible with the funding from this grant was executed as part of the LU-PRISM summer research program during the summer of 2014 with Kaitlyn Peters '16. In this project we measured the effect of leaf litter on the dynamics of nutrient cycling in the sediments from Lance Park Pond. The funds from the grant allowed us to purchase the reagents and lab equipment (e.g., BOD bottles) that we needed to complete this experiment. The results of this experiment are still being processed during this Fall (2014) but we are planning to present the results at the annual meetings of the Association of Southeastern Biologists and the Society for Freshwater Science. Additionally, this project allowed my lab to develop 2 collaborations with labs at other Universities. My lab is collaborating with Matthew Waters at Valsdosta State University in Georgia to measure the amount of carbon and nitrogen in the leaf samples from our experiment. Additionally we are collaborating with Vlad Gulis at Coastal Carolina University to measure the concentration of ergosterol (a proxy for fungi abundance) in the our leaf samples. 

\subsection{Future Plans}
The results of these three projects has defined the trajectory of the research for my lab in the near future.  First, these projects have resulted in samples that still need to be processed and will continue to provide research opportunities for Longwood students and additional data. Second, the results of these projects have yielded insights into the dynamics of small ponds that has raised a number of interesting new questions about the role of leaf litter in pond nutrient cycling and sediment metabolism. I plan to submit proposals to pursue these additional questions in the coming years.

\subsubsection{Project Completion Timeline}
\begin{tabular}{l l}
\toprule
\textbf{Objective} & \textbf{Target Completion Date} \\
\midrule
Complete Data Collection & Fall 2014 \\
Complete Data Analysis & Spring 2015 \\
Presentation of Findings & Spring and Summer 2015 \\
Manuscript Completion and Submission & Summer 2015 \\
\bottomrule
\end{tabular}

\section{Presentations with Longwood University Students}

\begin{itemize}
  \item L. Tacik, A. Choi, A. Gregoriou, C. Martin, K. McCusker, K. Peters, and K. Fortino. 2014. The decomposition of allochthonous detritus in man-made ponds in central Virginia. Poster at Society for Freshwater Science Annual Meeting, Portland, OR.
  \item L. Tacik, A. Choi, A. Gregoriou, C. Martin, K. Peters, and K. Fortino. 2014. The abundance and decomposition of coarse particulate organic matter (CPOM) in man-made ponds in central Virginia. Poster at Association of Southeastern Biologists Annual Meeting, Spartanburg, SC.
  \item L. Tacik, C. Martin, and K. Fortino. 2013. The distribution of coarse particulate organic matter (CPOM) in small ponds and its relationship to macroinvertebrate abundance and sediment organic matter. Poster at Cook-Cole Student Research Symposium, Longwood University, Farmville, VA.
  \item L. Tacik, C. Martin, and K. Fortino. 2013. The distribution of coarse particulate organic matter (CPOM) in small ponds and its relationship to macroinvertebrate abundance and sediment organic matter. Poster at Sigma-Xi Student Science Research Symposium, Hampden-Sydney College, Farmville, VA.
  \item K. Fortino. 2013. Aquatic Organic Matter Processing: The Biotic, the Abiotic, and a Changing Climate. Invited Talk, Virginia Commonwealth University. Richmond, VA.
\end{itemize}


\end{document}